\documentclass[a4paper, 12pt]{article}

\usepackage[italian]{babel}
\usepackage[utf8]{inputenc}
\usepackage[T1]{fontenc}
\usepackage{amssymb}
\usepackage{graphicx}
\usepackage{booktabs}
\usepackage{geometry}                
\geometry{letterpaper}   

\newcommand{\nl}{\\ & & }
\newcommand{\nr}{\\ \midrule}

\begin{document}

\begin{section}{A.A. 2013/2014}
\centering
\begin{tabular}{lll}
\toprule
Lezione & Data & Argomento\nr

1 & 14/1 & Richiami di matematica I: spazi di Hilbert \nl operatori hermitiani\nr
 
2 & 21/1 & Richiami II: classi di operatori, convergenza \nl Teo. spettrale generalizzato, Teo. di Stone\nl postulati della MQ, stati puri \nl principio di indeterminazione \nl misure proiettive \nl matrici densità\nr

3 & 23/1 & Sistemi compositi, stati entangled\nl matrice densità ridotta \nl evoluzione temporale in rapp. di Schrodinger e Heisenberg\nl serie di Dyson\nl Regole di quantizzazione \nr

4 & 28/1 & Quantizzazione canonica (Dirac)\nl Simmetrie e principio di relatività, teo. di Wigner\nl Particella libera 1D, rappresentazione canonica delle osservabili su $L^2(\mathbb{R})$ \nl pacchetti gaussiani, stati squeezed\nl Quasi-distribuzione di Wigner\nr

5 & 29/1 & Sistemi compositi (segue), traccia parziale, matrici densità ridotte.\nl
Paradosso EPR, disuguaglianza di Bell\nr

6 & 30/1 & Particella 1D (segue): equazione di Schrodinger\nl conservazione della corrente di probabilità\nl Oscillatore armonico: soluzione con gli operatori $a$, $a^\dagger$\nl stati di Fock, stati coerenti\nr

7 & 4/02 & Stati coerenti (segue), minima indeterminazione\nl operatori di displacement\nl Funzione caratteristica\nr

8 & 6/02 & Distribuzione di Wigner ottica\nl evoluzione temporale delle quasi distribuzioni\nl Algebre di Lie \nr


\end{tabular}

\begin{tabular}{lll}
Lezione & Data & Argomento\nr

9 & 11/2 & Algebre di Lie (segue) \nl Algebre di osservabili: momento angolare (SU(2)), two-mode harm. osc.\nl beam splitter, SU(1,1)\nr

10 & 13/2 & Particella 1D in un potenziale\nl discussione delle soluzioni all'eq. di Schrodinger\nl regioni classicamente accessibili e non\nl barriere e buche di potenziale\nr

11 & 18/2 & Approssimazione semiclassica (W.K.B.)\nl condizioni di raccordo\nl Riflessione e trasmissione di una barriera generica in WKB\nr

12 & 20/2 & Funzioni di Green\nl propagatore per la particella libera\nl Path integral\nl Campi magnetici e libertà di gauge\nl Trasformazioni di gauge in QM\nl Quantizzazione canonica per una particella spinless in campo magnetico\nr

13 & 25/2 & Spin in campo magnetico, matrici di Pauli\nl Effetto Aharonov-Bohm\nl Rappresentazione delle rotazioni\nl Definizione di vettori di operatori (matrici) \nr

14 & 27/2 & Rappresentazione di SO(3) sulle funzioni d'onda\nl momento angolare orbitale\nl Autovettori del mom. angolare\nl Modello di Schwinger\nl Spin e mom. angolare semi-intero: casi espliciti $J= 1/2, 1$\nr

14 & 27/2 & armoniche sferiche \nl somma di momenti angolari \nl tensori sferici, teorema di Wigner-Eckart \nr

14 & 26/2 & prodotto di tensori sferici \nl teorema della proiezione per vettori \nl particella in  potenziale centrale \nl [atomo di idrogeno risolto con l'eq. di Schrodinger]\nr

15 & 27/2 & particella carica con spin in potenziale centrale \nl fattore giromagnetico \nl accoppiamento spin-orbita \nl simmetrie discrete: parità \nr 

16 & 5/3 & simmetrie discrete: traslazioni discrete, teorema di Bloch \nl struttura a bande in un potenziale periodico (TB approx) \nl simmetrie discrete: time reversal per particelle spinless  \nr 

17 & 6/3 & time reversal con momento angolare e spin \nl teorema di Kramer \nl teoria perturbativa time indip. non degenere\nr 

18 & 12/3 & teoria perturbativa time-indip. degenere \nl risolvente di un operatore, legame tra quelli di $H$ e $H_0$\nr

\end{tabular}

\begin{tabular}{lll}
Lezione & Data & Argomento\nr

19 & 13/3 & teoria perturbativa con il formalismo dei risolventi \nl teoria perturbativa time-dip. \nl probabilità di assorbimento ed emissione stimolata \nl regola d'oro di Fermi \nl teorema adiabatico \nl fase di Berry \nr 

20 & 19/3 & fase di Berry e trasporto parallelo \nl particelle identiche e simmetrizzazione \nl spazi di hilbert fermionici e bosonici \nl effetti della simmetrizzazione sulla statistica \nr 

21 & 20/3 & interazione di scambio \nl particelle identiche localizzate sono distinguibili \nl seconda quantizzazione \nr 

22 & 25/3 & teorema di Wick \nl approccio time-indip. al problema dello Scattering \nl equazione di Lippmann-Schwinger, operatore di transfer \nl approssimazione di Born \nr

23 & 26/3 & validità dell'approssimazione di Born \nl teorema ottico \nl scattering con d.o.f. interni \nl matrice di Scattering \nl simmetrie dell'ampiezza di scattering \nl approccio time-dip. allo Scattering  \nr
\bottomrule
\end{tabular}

\end{section}


\end{document}
