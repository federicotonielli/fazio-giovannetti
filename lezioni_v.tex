AA 2012/2013


LEZ 1 (8/1):  richiami di algebra lineare, di analisi funzionale (spazi di hilbert)

LEZ 2 (10/1): richiami di analisi funzionale (seroe di fourier, norme operatoriali, continuità, polar dec, proprietà degli operatori, teoremi spettrali)

LEZ 3 (15/1): autovalori/vettori generalizzati e rigged hilbert space, [postulati saltati], operatore posizione, matrice densità, disug. di robertson e principio di indeterminazione

LEZ 4 (17/1): matrice densità ridotta, quantizzazione canonica, quantizzazione definendo gli operatori a partire dalle simmetrie della teoria, evoluzione temporale in MQ, serie di Dyson

LEZ 5 (22/1): rappresentazioni di schrodinger, heisenberg e di interazione, teorema di wigner, proprietà generiche di un sistema quantistico: conservazione della corrente e teorema di eherenfest, oscillatore armonico con gli operatori di creazione e distruzione

LEZ 6 (24/1): discussione sulla natura discreta/continua dello spettro in relazione all'andamento del potenziale, quasi-distribuzione di Wigner, operatore di displacement

LEZ 7 (29/1): distribuzione di Wigner per gli stati dell'oscillatore armonico: decomposizione in stati coerenti, calcolo di W con gli op. di displacement, 
