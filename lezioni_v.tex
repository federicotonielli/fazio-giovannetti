\documentclass[a4paper, 12pt]{article}

\usepackage[italian]{babel}
\usepackage[utf8]{inputenc}
\usepackage[T1]{fontenc}
\usepackage{graphicx}
\usepackage{booktabs}
\usepackage{geometry}                
\geometry{letterpaper}   

\newcommand{\nl}{\\ & & }
\newcommand{\nr}{\\ \midrule}

\begin{document}
\title{
    \textbf{Registri delle lezioni del corso interno di MQ} \\[0.5 cm]
    A.A. 2012/2013 \\[0.2 cm] 
    A.A. 2013/2014}
\date{}

\maketitle

\begin{section}{A.A. 2012/2013}
\begin{tabular}{lll}
\toprule
Lezione & Data & Argomenti\nr

1 & 8/1 & richiami di algebra lineare e analisi funzionle\nr
 
2 & 10/1 & richiami di algebra lineare e analisi funzionale \nl operatori posizione ed impulso\nr

3 & 15/1 & autovettori/valori generalizzati e rigged hilbert space \nl [postulati della MQ] \nl matrice densità \nl principio di indeterminazione\nr

4 & 17/1 & matrice densità ridotta \nl quantizzazione a partire dalle simmetrie \nl quantizzazione canonica \nl evoluzione temporale e serie di Dyson\nr

5 & 22/1 & rappresentazioni di Schrodinger, di Heisenberg, di interazione \nl fatti generali: conservazione della corrente di probabilità \nl fatti generali: teorema di Eherenfest \nl oscilaatore armonico risolto con $a$ e $a^+$ \nl [oscillatore armonico risolto con l'eq. di Schrodinger] \nr

6 & 24/1 & discussione sulla natura discreta/continua dello spettro\nl quasi-distribuzione di Wigner \nl stati coerenti: operatore di displacement \nr

7 & 29/1 & stati coerenti: insieme overcompleto \nl calcolo della funzione di Wigner per l'osc. armonico \nl stati che saturano il principio di indeterminazione \nr

8 & 31/1 & algebre di Lie \nl effetto tunnel e matrici di trasferimento \nl approssimazione WKB: soluzioni \nr

9 & 5/2 & approssimazione WKB: raccordo delle soluzioni \nl quantizzazione di Bohr-Sommerfeld \nl propagatore: calcolo nel caso di particella libera \nl propagatore: espressione formale come path integral \nr

\end{tabular}

\begin{tabular}{lll}
Lezione & Data & Argomenti\nr

10 & 7/2 & particella carica spinless in campo magnetico esterno \nl particella carica con spin in campo magnetico esterno \nr

11 & 12/2 & effetto Aharonov-Bohm \nl teorema di Wigner \nl struttura di SO(3), def. di vettore \nr 

12 & 19/2 & rappresentazione generica delle rotazioni, momento angolare \nl modello di Schwinger \nl casi espliciti $J=1/2,1$ \nl momento angolare orbitale  \nr 

13 & 20/2 & armoniche sferiche \nl somma di momenti angolari \nl tensori sferici, teorema di Wigner-Eckart \nr

14 & 26/2 & prodotto di tensori sferici \nl teorema della proiezione per vettori \nl particella in  potenziale centrale \nl [atomo di idrogeno risolto con l'eq. di Schrodinger]\nr

15 & 27/2 & particella carica con spin in potenziale centrale \nl fattore giromagnetico \nl accoppiamento spin-orbita \nl simmetrie discrete: parità \nr 

16 & 5/3 & simmetrie discrete: traslazioni discrete, teorema di Bloch \nl struttura a bande in un potenziale periodico (TB approx) \nl simmetrie discrete: time reversal per particelle spinless  \nr 

17 & 6/3 & time reversal per particelle con spin \nl teorema di Kramer \nl teoria perturbativa time indip. non degenere\nr 

18 & 12/3 & teoria perturbativa time-indip. degenere \nl risolvente di un operatore, legame tra quelli di $H$ e $H_0$\nr

\end{tabular}

\begin{tabular}{lll}
Lezione & Data & Argomenti\nr

19 & 13/3 & teoria perturbativa con il formalismo dei risolventi \nl teoria perturbativa time-dip. \nl probabilità di assorbimento ed emissione stimolata \nl regola d'oro di Fermi \nl teorema adiabatico \nl fase di Berry \nr 

20 & 19/3 & fase di Berry e trasporto parallelo \nl particelle identiche e simmetrizzazione \nl spazi di hilbert fermionici e bosonici \nl effetti della simmetrizzazione sulla statistica \nr 

21 & 20/3 & interazione di scambio \nl particelle identiche localizzate sono distinguibili \nl seconda quantizzazione \nr 

22 & 25/3 & teorema di Wick \nl approccio time-indip. al problema dello Scattering \nl equazione di Lippmann-Schwinger, operatore di transfer \nl approssimazione di Born \nr

23 & 26/3 & validità dell'approssimazione di Born \nl teorema ottico \nl scattering con d.o.f. interni \nl matrice di Scattering \nl effetto delle simmetrie sull'ampiezza di scattering \nl approccio time-dip. allo Scattering  \nr
\bottomrule
\end{tabular}

\end{section}

\newpage

\begin{section}{A.A. 2013/2014}

 \begin{tabular}{lll}
  \toprule

  Lezione & Data & Argomento\nr
  
  1 &  & richiami di algebra lineare ed analisi funzionale \nr
  2 &  & 





 \end{tabular}

\end{section}


\end{document}
