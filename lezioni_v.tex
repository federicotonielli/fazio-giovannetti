\documentclass[a4paper, 12pt]{article}

\usepackage[italian]{babel}
\usepackage[utf8]{inputenc}
\usepackage[T1]{fontenc}
\usepackage{graphicx}
\usepackage{booktabs}
\usepackage{geometry}                
\geometry{letterpaper}   

\newcommand{\nl}{\\ & & }
\newcommand{\nr}{\\ \midrule}

\begin{document}
 

\begin{section}{A.A. 2012/2013}
\begin{tabular}{lll}
\toprule
Lezione & Data & Argomento\nr

1 & 8/1 & richiami di algebra lineare e analisi funzionle\nr
 
2 & 10/1 & richiami di algebra lineare e analisi funzionale \nl operatori posizione ed impulso\nr

3 & 15/1 & autovettori/valori generalizzati e rigged hilbert space \nl [postulati della MQ] \nl matrice densità \nl principio di indeterminazione\nr

4 & 17/1 & matrice densità ridotta \nl quantizzazione con gli operatori di simmetria \nl quantizzazione canonica \nl evoluzione temporale e serie di Dyson\nr

5 & 22/1 & rappresentazioni di Schrodinger, di Heisenberg, di interazione \nl fatti generali: conservazione della corrente di probabilità \nl fatti generali: teorema di Eherenfest \nl oscilaatore armonico risolto con $a$ e $a^+$ \nr

6 & 24/1 & discussione sulla natura discreta/continua dello spettro\nl quasi-distribuzione di Wigner \nl stati coerenti: operatore di displacement \nr

7 & 29/1 & stati coerenti: insieme overcompleto \nl calcolo della funzione di Wigner per l'osc. armonico \nl stati che saturano il principio di indeterminazione \nr

8 & 31/1 & algebre di Lie \nl effetto tunnel e matrici di trasferimento \nl approssimazione WKB: soluzioni \nr

9 & 5/2 & approssimazione WKB: raccordo delle soluzioni \nl quantizzazione di Bohr-Sommerfeld \nl propagatore: calcolo nel caso di particella libera \nl propagatore: espressione formale come path integral \nr

10 & 7/2 & particella carica spinless in campo magnetico esterno \nl particella carica con spin in campo magnetico esterno \nr

\end{tabular}

\begin{tabular}{lll}
Lezione & Data & Argomento\nr

11 & 12/2 & effetto Aharonov-Bohm \nl teorema di Wigner \nl struttura di SO(3), def. di vettore \nr 

12 & 19/2 & rappresentazione generica delle rotazioni, momento angolare \nl modello di Schwinger \nl casi espliciti $J=1/2,1$ \nl momento angolare orbitale  \nr 

13 & 20/2 & armoniche sferiche \nl somma di momenti angolari \nl tensori sferici, teorema di Wigner-Eckart \nr

14 & 26/2 & prodotto di tensori sferici \nl teorema della proiezione per vettori \nl particella in  potenziale centrale \nr

15 & 27/2 & particella carica con spin in potenziale centrale \nl fattore giromagnetico \nl accoppiamento spin-orbita \nl simmetrie discrete: parità \nr 

16 & 
\bottomrule
\end{tabular}


\end{section}


\end{document}
